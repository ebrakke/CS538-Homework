\documentclass[11pt]{article}
\usepackage{amssymb,amsmath,amsthm,graphicx}
\usepackage{fancyhdr}

\def\shownotes{1}   % set 1 for version with author notes
                    % set 0 for no notes



%uncomment to get hyperlinks
%\usepackage{hyperref}

%%%%%%%%%%%%%%%%%%%%%%%%%%%%%%%%%%%%%%%%%%%%%%%%%%%%%%%%%%%%%%
%Some macros (you can ignore everything until "end of macros")

\topmargin 0pt \advance \topmargin by -\headheight \advance
\topmargin by -\headsep

\textheight 8.9in

\oddsidemargin 0pt \evensidemargin \oddsidemargin \marginparwidth
0.5in

\textwidth 6.5in

%%%%%%

\providecommand{\vs}{vs. }
\providecommand{\ie}{\emph{i.e.,} }
\providecommand{\eg}{\emph{e.g.,} }
\providecommand{\cf}{\emph{cf.,} }
\providecommand{\etc}{\emph{etc.} }

\newcommand{\getsr}{\gets_{\mbox{\tiny R}}}
\newcommand{\bits}{\{0,1\}}
\newcommand{\bit}{\{0,1\}}
\newcommand{\Ex}{\mathbb{E}}
\newcommand{\eqdef}{\stackrel{def}{=}}
\newcommand{\To}{\rightarrow}
\newcommand{\e}{\epsilon}
\newcommand{\R}{\mathbb{R}}
\newcommand{\N}{\mathbb{N}}
\newcommand{\Gen}{\mathsf{Gen}}
\newcommand{\Enc}{\mathsf{Enc}}
\newcommand{\Dec}{\mathsf{Dec}}
\newcommand{\Sign}{\mathsf{Sign}}
\newcommand{\Ver}{\mathsf{Ver}}

\providecommand{\mypara}[1]{\smallskip\noindent\emph{#1} }
\providecommand{\myparab}[1]{\smallskip\noindent\textbf{#1} }
\providecommand{\myparasc}[1]{\smallskip\noindent\textsc{#1} }
\providecommand{\para}{\smallskip\noindent}


\newtheorem{theorem}{Theorem}
\theoremstyle{definition}
\newtheorem{ex}{Exercise}
\newtheorem{definition}{Definition}

%%%%%%%  Author Notes %%%%%%%d
%
\ifnum\shownotes=1
\newcommand{\authnote}[2]{{ $\ll$\textsf{\footnotesize #1 notes: #2}$\gg$}}
\else
\newcommand{\authnote}[2]{}
\fi
\newcommand{\Snote}[1]{{\authnote{Solution}{#1}}}
\newcommand{\Inote}[1]{{\authnote{Solution}{#1}}}
\newcommand{\Ichanged}[1]{{\authnote{Changed}{#1}}}
%%%%%%%%%%%%%%%%%%%%%%%%%%%%%%%%%

\newcommand{\VAR}{\mathrm{VAR}}



% end of macros
%%%%%%%%%%%%%%%%%%%%%%%%%%%%%%%%%%%%%%%%%%%%%%%%%%%%%%%%%%%%%%


% page counting, header/footer
\usepackage{fancyhdr}
\pagestyle{fancy}
\lhead{\footnotesize \parbox{11cm}{CS538, Boston University, Fall 2015} }
\rhead{Erik Brakke}
\renewcommand{\headheight}{24pt}

\begin{document}

\title{Homework 5}
\author{Erik Brakke}
\maketitle

\thispagestyle{fancy}

\myparab{Collaborators: }  Alison Kendler, Kyle Hogan, Isaac Cohen.
 
 
\section*{Answer 1}
Proof: We know that $f$ is a permutation\\
Thus, we know that the last $2k$ bits of $f'$ are a permutation on $\{0,1\}^{2k}$\\
The first $k$ bits are all independent of each other, making the total possibilities of the first $k$ bits $2^k$\\
This means that $f'$ has $2^{3k}$ possible outputs\\
$x$ always has an output for $f$, and $c$ can always be found for $x$\\
This means that any $x$ of length $3k$ always has an output for $f'$\\
For $x \neq y$, either $f(x) \neq f(y)$ or $c(s_x) \neq c(s_y)$\\
It is possible for the minority bits to all be in the same positions, but if $x \neq y$, then the majority bit must be different\\
It is possible for the majority bits to all be the same, but if $x \neq y$, then the minority bit must be in a different position\\
This means that $f'(x)$ has a unique output in $\{0,1\}^{3k}$\\
This means that $f'$ is a permutation\\
\newline
Now I will prove the one-wayness of $f'$ with a reduction\\
Let's assume that $f'$ was not a one-way function.  This means that there is an algorithm $D$ such that given $f'(x)$, $D$ could compute $x$ in polynomial-time\\
We can use the fact that we can reverse $f'$ to reverse $f$\\
Given $f'(s)$ we can use $D$ to find $s$.\\
Once we have $s$, we can compute easily compute $d(s_1) \circ ... \circ d(s_k)$\\
These are the inputs to $f$, meaning that using $D$ we can reverse $f$\\
However we assumed that $f$ was a one way functions, therefore $f'$ must be a one way function as well\\
\qed

Given a bit position $i$, one can compute $i//3$ to figure out $c$ by looking at the $i//3$ position of $f'(x)$.  This is the majority bit of the group of 3 that $i$ is in\\
If the majority bit is $1$, then $\Pr[i = 1] = 3/4$ because out of the 4 possible combinations where $c = 1$, 1 appears 9 times out of the $12$ bits\\
The same logic holds for $c = 0$\\
Because computing $i//3$ runs in polynomial time, and worst case you would have to search $k$ positions, the algorithm runs in linear time\\

\section*{Answer 2}
We can use CRT to compute $c_{DB} = m^2 mod (n_d * n_b)$\\
We know that if $m < \sqrt{n}$, then given $c$, m is just the integer square root of $c$\\
$m$ must be less than $\sqrt{n_d * n_b}$ because $m < min(n_d,n_b)$\\
$min(n_d,n_b) < \sqrt(n_d * n_b)$\\
So now we can just take the integer square root of $c_{DB}$ to get $m$\\

\section*{Answer 3}
Let's assume we have an algorithm $R$ that can predict $SK$ with non negligible probability\\
Now we can use $R$ to build a distinguisher $D$ that can break the definition of secure multi-bit encryption\\
$D$ chooses two messages $m_0,m_1$ after seeing $PK$\\
$D$ then uses $R$ to compute $SK$ from $PK$ with non negligible probability\\
When $D$ is presented with $Enc_{PK}(m)$, he can run $Dec_{SK}(Enc_{PK}(m))$ to obtain $m$\\
He can then distinguish between $m_0,m_1$ with non-negligible probability\\
This contradicts our assumption that (Gen,Enc,Dec) is a secure public key crypto system\\
Therefore, such an $R$ cannot exist\\
\qed


\noindent\hrulefill


\section*{References}

None

\end{document} 