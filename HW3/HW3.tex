\documentclass[11pt]{article}
\usepackage{amssymb,amsmath,amsthm,url,graphicx}
\usepackage{fancyhdr}

\def\shownotes{1}   % set 1 for version with author notes
                    % set 0 for no notes



%uncomment to get hyperlinks
%\usepackage{hyperref}

%%%%%%%%%%%%%%%%%%%%%%%%%%%%%%%%%%%%%%%%%%%%%%%%%%%%%%%%%%%%%%
%Some macros (you can ignore everything until "end of macros")

\topmargin 0pt \advance \topmargin by -\headheight \advance
\topmargin by -\headsep

\textheight 8.9in

\oddsidemargin 0pt \evensidemargin \oddsidemargin \marginparwidth
0.5in

\textwidth 6.5in

%%%%%%

\providecommand{\vs}{vs. }
\providecommand{\ie}{\emph{i.e.,} }
\providecommand{\eg}{\emph{e.g.,} }
\providecommand{\cf}{\emph{cf.,} }
\providecommand{\etc}{\emph{etc.} }

\newcommand{\getsr}{\gets_{\mbox{\tiny R}}}
\newcommand{\bits}{\{0,1\}}
\newcommand{\bit}{\{0,1\}}
\newcommand{\Ex}{\mathbb{E}}
\newcommand{\eqdef}{\stackrel{def}{=}}
\newcommand{\To}{\rightarrow}
\newcommand{\e}{\epsilon}
\newcommand{\R}{\mathbb{R}}
\newcommand{\N}{\mathbb{N}}
\newcommand{\Gen}{\mathsf{Gen}}
\newcommand{\Enc}{\mathsf{Enc}}
\newcommand{\Dec}{\mathsf{Dec}}
\newcommand{\Sign}{\mathsf{Sign}}
\newcommand{\Ver}{\mathsf{Ver}}

\providecommand{\mypara}[1]{\smallskip\noindent\emph{#1} }
\providecommand{\myparab}[1]{\smallskip\noindent\textbf{#1} }
\providecommand{\myparasc}[1]{\smallskip\noindent\textsc{#1} }
\providecommand{\para}{\smallskip\noindent}


\newtheorem{theorem}{Theorem}
\theoremstyle{definition}
\newtheorem{ex}{Exercise}
\newtheorem{definition}{Definition}

%%%%%%%  Author Notes %%%%%%%d
%
\ifnum\shownotes=1
\newcommand{\authnote}[2]{{ $\ll$\textsf{\footnotesize #1 notes: #2}$\gg$}}
\else
\newcommand{\authnote}[2]{}
\fi
\newcommand{\Snote}[1]{{\authnote{Solution}{#1}}}
\newcommand{\Inote}[1]{{\authnote{Solution}{#1}}}
\newcommand{\Ichanged}[1]{{\authnote{Changed}{#1}}}
%%%%%%%%%%%%%%%%%%%%%%%%%%%%%%%%%

\newcommand{\VAR}{\mathrm{VAR}}



% end of macros
%%%%%%%%%%%%%%%%%%%%%%%%%%%%%%%%%%%%%%%%%%%%%%%%%%%%%%%%%%%%%%


% page counting, header/footer
\usepackage{fancyhdr}
\usepackage{lastpage}
\pagestyle{fancy}
\lhead{\footnotesize \parbox{11cm}{CS538, Boston University, Fall 2015} }
\rhead{Erik Brakke}
\renewcommand{\headheight}{24pt}

\begin{document}

\title{Homework 3}
\author{Erik Brakke}
\maketitle

\thispagestyle{fancy}

\myparab{Collaborators: }  .
 
 
\section*{Answer 1}
$G'$ is not a pseudo random generator.\\
Proof: Assume that $G'$ is a pseudo random generator.\\
This means that it is Next Bit Unpredictable (Def'n of a PRG)\\
In other words, there exists an algorithm $A$ such that $\Pr[A(b_0...b_i) = b_{i+1}] > 1/2 + n(x)$\\
However, such an algorithm for $A$ exists such that $\Pr[A(b_0...b_i) = b_{i+1}] = 1$\\
Assume we have a seed $s$ of length $L$.  $G(s)'$ is then computed\\
$A$ gets $b_0$.  It knows this is the xor of all the bits in $w$\\
$A$ then asks for bits $b_1...b_{L-1}$ and xors them all together to get some bit $x$\\
If $x = b_0 = 0$ then $b_L = 1$, if $x = b_0 = 1$ then $b_L = 0$,if $x != b_0$ then $b_L = \neg b_{L-1}$\\
Because $A$ runs in poly time and can always predict the last bit, $G'$ is not next bit unpredictable and is therefore not a PRG\\
\qed

\section*{Answer 2}
\begin{enumerate}
	\item[(a)]
	Proof: Assume that $\bar{G}$ is not a pseudo random generator\\
	This means that there exists an efficent algorithm $A$ such that $\Pr[A(\bar{G}(s) = \text{ PRG })] - \Pr[A(R(s)) = \text{ PRG }] = f(x)$ Where $R(s)$ is a truely random output and $f$ is non-negligible  (Def'n of indistinguishability of PRG)\\
	We can use $A$ to then detect whether or not $G(s)$ is random or not.\\
	Negate every bit of $G(s)$ and feed it to $A$.\\
	If $A(\neg G(s)) =$ PRG, then we can say with a non-negligible probability that $G(s)$ is a PRG, and likewise if $A$ outputs 'random'.\\
	However, we assumed that $G$ was a PRG, therefore we should not have an algorithm that can be run in poly-time that will distinguish its output from random\\
	Therefore, we have a contradiction and $\bar{G}$ is a PRG\\
	\qed

	\item[(b)]
\end{enumerate}

\section*{Answer 3}
\begin{enumerate}
	\item[(a)]
	We want to find which $s1,s2$ is equivalent to $g^{x/2} (The principle root)\\
	Becuase we know that $g^x < p$ we can also say that $g^{x/2} < p/2$\\
	We know that one of $s_1,s_2$ is the principle root, so if $D(s_1) = 1$ then output $s_1$ else output $s_2$\\
	$\Pr[D(s_1) = s_1 | s_1 \text{ is the principle bit}] = 1/2 + \epsilon$\\
	Therefore with non-negligible probability, we can use $D$ to which roo is the principle root\\
\end{enumerate}
	




\noindent\hrulefill


\section*{References}

None

\end{document} 