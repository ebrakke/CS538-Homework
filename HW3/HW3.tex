\documentclass[11pt]{article}
\usepackage{amssymb,amsmath,amsthm,url,graphicx}
\usepackage{fancyhdr}

\def\shownotes{1}   % set 1 for version with author notes
                    % set 0 for no notes



%uncomment to get hyperlinks
%\usepackage{hyperref}

%%%%%%%%%%%%%%%%%%%%%%%%%%%%%%%%%%%%%%%%%%%%%%%%%%%%%%%%%%%%%%
%Some macros (you can ignore everything until "end of macros")

\topmargin 0pt \advance \topmargin by -\headheight \advance
\topmargin by -\headsep

\textheight 8.9in

\oddsidemargin 0pt \evensidemargin \oddsidemargin \marginparwidth
0.5in

\textwidth 6.5in

%%%%%%

\providecommand{\vs}{vs. }
\providecommand{\ie}{\emph{i.e.,} }
\providecommand{\eg}{\emph{e.g.,} }
\providecommand{\cf}{\emph{cf.,} }
\providecommand{\etc}{\emph{etc.} }

\newcommand{\getsr}{\gets_{\mbox{\tiny R}}}
\newcommand{\bits}{\{0,1\}}
\newcommand{\bit}{\{0,1\}}
\newcommand{\Ex}{\mathbb{E}}
\newcommand{\eqdef}{\stackrel{def}{=}}
\newcommand{\To}{\rightarrow}
\newcommand{\e}{\epsilon}
\newcommand{\R}{\mathbb{R}}
\newcommand{\N}{\mathbb{N}}
\newcommand{\Gen}{\mathsf{Gen}}
\newcommand{\Enc}{\mathsf{Enc}}
\newcommand{\Dec}{\mathsf{Dec}}
\newcommand{\Sign}{\mathsf{Sign}}
\newcommand{\Ver}{\mathsf{Ver}}

\providecommand{\mypara}[1]{\smallskip\noindent\emph{#1} }
\providecommand{\myparab}[1]{\smallskip\noindent\textbf{#1} }
\providecommand{\myparasc}[1]{\smallskip\noindent\textsc{#1} }
\providecommand{\para}{\smallskip\noindent}


\newtheorem{theorem}{Theorem}
\theoremstyle{definition}
\newtheorem{ex}{Exercise}
\newtheorem{definition}{Definition}

%%%%%%%  Author Notes %%%%%%%d
%
\ifnum\shownotes=1
\newcommand{\authnote}[2]{{ $\ll$\textsf{\footnotesize #1 notes: #2}$\gg$}}
\else
\newcommand{\authnote}[2]{}
\fi
\newcommand{\Snote}[1]{{\authnote{Solution}{#1}}}
\newcommand{\Inote}[1]{{\authnote{Solution}{#1}}}
\newcommand{\Ichanged}[1]{{\authnote{Changed}{#1}}}
%%%%%%%%%%%%%%%%%%%%%%%%%%%%%%%%%

\newcommand{\VAR}{\mathrm{VAR}}



% end of macros
%%%%%%%%%%%%%%%%%%%%%%%%%%%%%%%%%%%%%%%%%%%%%%%%%%%%%%%%%%%%%%


% page counting, header/footer
\usepackage{fancyhdr}
\usepackage{lastpage}
\pagestyle{fancy}
\lhead{\footnotesize \parbox{11cm}{CS538, Boston University, Fall 2015} }
\rhead{Erik Brakke}
\renewcommand{\headheight}{24pt}

\begin{document}

\title{Homework 3}
\author{Erik Brakke}
\maketitle

\thispagestyle{fancy}

\myparab{Collaborators: } Alison Kendler, Kyle Hogan, Issac Cohen
 
 
\section*{Answer 1}
$G'$ is not a pseudo random generator.\\
Proof: Assume that $G'$ is a pseudo random generator.\\
This means that it is Next Bit Unpredictable (Def'n of a PRG)\\
In other words, there exists an algorithm $A$ such that $\Pr[A(b_0...b_i) = b_{i+1}] > 1/2 + n(x)$\\
However, such an algorithm for $A$ exists such that $\Pr[A(b_0...b_i) = b_{i+1}] = 1$\\
Assume we have a seed $s$ of length $L$.  $G(s)'$ is then computed\\
$A$ gets $b_0$.  It knows this is the xor of all the bits in $w$\\
$A$ then asks for bits $b_1...b_{L-1}$ and xors them all together to get some bit $x$\\
If $x = b_0 = 0$ then $b_L = 1$, if $x = b_0 = 1$ then $b_L = 0$,if $x != b_0$ then $b_L = \neg b_{L-1}$\\
Because $A$ runs in poly time and can always predict the last bit, $G'$ is not next bit unpredictable and is therefore not a PRG\\
\qed

\section*{Answer 2}
\begin{enumerate}
	\item[(a)]
	Proof: Assume that $\bar{G}$ is not a pseudo random generator\\
	This means that there exists an efficent algorithm $A$ such that $\Pr[A(\bar{G}(s) = \text{ PRG })] - \Pr[A(R(s)) = \text{ PRG }] = f(x)$ Where $R(s)$ is a truely random output and $f$ is non-negligible  (Def'n of indistinguishability of PRG)\\
	We can use $A$ to then detect whether or not $G(s)$ is random or not.\\
	Negate every bit of $G(s)$ and feed it to $A$.\\
	If $A(\neg G(s)) =$ PRG, then we can say with a non-negligible probability that $G(s)$ is a PRG, and likewise if $A$ outputs 'random'.\\
	However, we assumed that $G$ was a PRG, therefore we should not have an algorithm that can be run in poly-time that will distinguish its output from random\\
	Therefore, we have a contradiction and $\bar{G}$ is a PRG\\
	\qed

	\item[(b)]
	Proof: Let's assume that $G_2$ negates every bit of $G_1$\\ We know that these are both pseudo random generators by the proof in part (a)\\
	However, Once $G_3$ is run, it becomes obvious that the second half of the message is just the flipped bits of the first half.\\
	We can design an algorithm $A(G(s))$ such that it can split the message in half and xor all of the bits.  $A$ outputs $k$ 1s, where $k$ is the length of the seed, then $A$ knows that $G(s)$ is pseudo random.  This is a polynomial time alogrithm, and will always be right if $G_2$ is just the negation of $G_1$\\
	\qed
\end{enumerate}

\section*{Answer 3}
\begin{enumerate}
	\item[(a)]
	We want to find which $s1,s2$ is equivalent to $g^{x/2}$ (The principle root)\\
	Becuase we know that $g^x < p$ we can also say that $g^{x/2} < p/2$\\
	We know that one of $s_1,s_2$ is the principle root, so if $D(s_1) = 1$ then output $s_1$ else output $s_2$\\
	$\Pr_y[D(s_1) = s_1 | s_1 \text{ is the principle bit}] = 1/2 + \epsilon$\\
	Therefore with non-negligible probability, we can use $D$ to which root is the principle root\\

	\item[(b)]
	Proof: We want to show that $\Pr[\text{Principle root of z} \neq \text{Principle root of y} * g^r] \le \frac{1}{t}$\\
	We know $z \equiv yg^{2r}$, $y \equiv g^x$\\
	$z \equiv g^{x + 2r}$\\
	So we want to show when $g^{(x + 2r)/2} \neq g^{x/2} * g^r$\\
	This will be the case when $(x + 2r)/2 \neq x/2 + r$\\
	We know that $1 \le x \le (p-1)/t$ and is even and $1 \le r \le (p-1)/2$, so $x/2 \le (p-1)/2t$ and $r \le (p-1)/2$\\
	However, $x + 2r$ can be $> (p-1)$.  If this is the case, then $(x + 2r)/2$ will no longer actually be equal to $(x/2) + r$.\\
	This is the case when $g^{(x + 2r)/2} \neq g^{x/2} * g^r$\\
	So we can reduce our problem to $\Pr[x + 2r < (p-1)]$\\
	Let's consider the worst case, where $x = (p-1)/t$\\
	$\Pr[(p-1)/t + 2r > (p-1)] = \Pr[r > (p-1)/2 - (p-1)/2t]$\\
	$1 - \frac{(p-1)/2 - (p-1)/2t}{(p-1)/2}$ (by using counting principles)\\
	$1 - (1 - 1/t) = 1/t$\\
	Therefore, $\Pr_r[\text{Principle(z)} \neq \text{Principle(y) * g^r}] \le 1/t$

	\item[(c)]
	Proof: We know that we have an algorithm $D(s_1,s_2,p,g,y) -> \text{Princple square root}$ with probability $1/2 + \episolon$ (Part (a))\\
	We also know that $\Pr[Principle(z) \neq Principle(y) * g^r] \le 1/t$\\
	Using this information we can construct an algorithm $A$ that gives us the principle square root of $y$ with non-negligible probability\\
	We can now try to make sure that $y$ is distributed uniformly in $(p-1)$ by taking some random bit $r$ such that $1 \le r \le (p-1)/2$ and taking $yg^{2r} = z$ (From the previous part)\\
	We can split $y$ into it's two roots efficiently (given in the problem)\\
	We can then run $D$ on one of the roots of $y$ to see which one the princple square root is.\\
	So now we have that $Principle(z) = D(y) * g^r$\\
	This is right with $\Pr = 1/2 + \epsilon - 1/t$ because we have a $1/2 + \epsilon$ chance that $D(y)$ is correct, and we know that there is only a $1\t$ chance that the principle square root of $z$ is not equal to the principle square root of $y * g^r$ (from part (b))\\
	Therefore, We can use $D$ to find the principle square root of $y$ with $\Pr = 1/2 + \epsilon - 1/t$\\
	\qed

	\item[(d)]
	Using Heoffding's bound, we want to bound the probability that the majority vote is wrong\\
	We have our algorithm $A$ that will find the principle square root of any $y$ in $t-IS$ with probability $1/2 + \epsilon - 1/t$\\
	Therefore we want to show the bound of $\Pr[E[A_n] - Actual[A] \ge \epsilon - 1/t]$\\
	We can use the Hoeffding's to show that this is $\le e^{-2(\epsilon - 1/t)^2n}$\\
	This is an exponentially small number, and as we run many trials, this number becomes negligible\\
	Therefore, The probability that we don't find the principle square root of $y$ in $t-IS$ using $A$ after many trials is negligible\\



\end{enumerate}


\noindent\hrulefill


\section*{References}

None

\end{document} 