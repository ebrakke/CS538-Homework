\documentclass[11pt]{article}
\usepackage{amssymb,amsmath,amsthm,url,graphicx}
\usepackage{fancyhdr}

\def\shownotes{1}   % set 1 for version with author notes
                    % set 0 for no notes



%uncomment to get hyperlinks
%\usepackage{hyperref}

%%%%%%%%%%%%%%%%%%%%%%%%%%%%%%%%%%%%%%%%%%%%%%%%%%%%%%%%%%%%%%
%Some macros (you can ignore everything until "end of macros")

\topmargin 0pt \advance \topmargin by -\headheight \advance
\topmargin by -\headsep

\textheight 8.9in

\oddsidemargin 0pt \evensidemargin \oddsidemargin \marginparwidth
0.5in

\textwidth 6.5in

%%%%%%

\providecommand{\vs}{vs. }
\providecommand{\ie}{\emph{i.e.,} }
\providecommand{\eg}{\emph{e.g.,} }
\providecommand{\cf}{\emph{cf.,} }
\providecommand{\etc}{\emph{etc.} }

\newcommand{\getsr}{\gets_{\mbox{\tiny R}}}
\newcommand{\bits}{\{0,1\}}
\newcommand{\bit}{\{0,1\}}
\newcommand{\Ex}{\mathbb{E}}
\newcommand{\eqdef}{\stackrel{def}{=}}
\newcommand{\To}{\rightarrow}
\newcommand{\e}{\epsilon}
\newcommand{\R}{\mathbb{R}}
\newcommand{\N}{\mathbb{N}}
\newcommand{\Gen}{\mathsf{Gen}}
\newcommand{\Enc}{\mathsf{Enc}}
\newcommand{\Dec}{\mathsf{Dec}}
\newcommand{\Sign}{\mathsf{Sign}}
\newcommand{\Ver}{\mathsf{Ver}}

\providecommand{\mypara}[1]{\smallskip\noindent\emph{#1} }
\providecommand{\myparab}[1]{\smallskip\noindent\textbf{#1} }
\providecommand{\myparasc}[1]{\smallskip\noindent\textsc{#1} }
\providecommand{\para}{\smallskip\noindent}


\newtheorem{theorem}{Theorem}
\theoremstyle{definition}
\newtheorem{ex}{Exercise}
\newtheorem{definition}{Definition}

%%%%%%%  Author Notes %%%%%%%d
%
\ifnum\shownotes=1
\newcommand{\authnote}[2]{{ $\ll$\textsf{\footnotesize #1 notes: #2}$\gg$}}
\else
\newcommand{\authnote}[2]{}
\fi
\newcommand{\Snote}[1]{{\authnote{Solution}{#1}}}
\newcommand{\Inote}[1]{{\authnote{Solution}{#1}}}
\newcommand{\Ichanged}[1]{{\authnote{Changed}{#1}}}
%%%%%%%%%%%%%%%%%%%%%%%%%%%%%%%%%

\newcommand{\VAR}{\mathrm{VAR}}



% end of macros
%%%%%%%%%%%%%%%%%%%%%%%%%%%%%%%%%%%%%%%%%%%%%%%%%%%%%%%%%%%%%%


% page counting, header/footer
\usepackage{fancyhdr}
\usepackage{lastpage}
\pagestyle{fancy}
\lhead{\footnotesize \parbox{11cm}{CS538, Boston University, Fall 2015} }
\rhead{Erik Brakke}
\renewcommand{\headheight}{24pt}

\begin{document}

\title{Homework 2}
\author{Erik Brakke}
\maketitle

\thispagestyle{fancy}

\myparab{Collaborators: }  .
 
 
\section*{Answer 1}
\begin{enumerate}
	\item[(a)]
	Thm: If $x \equiv y$ (mod $p-1$) then for any $a, a^x \equiv a^y$ (mod $p$)\\
	\newline
	Proof:  $x = (p-1)k_x + r$ and $y = (p-1)k_y + r$ (by the unique fact about division)\\
	Therefore, $a^{(p-1)k_x + r} \equiv a^{(p-1)k_y + r}$ (mod $p$) (using substitution)\\
	$a^{(p-1)k_x + r} \text{ mod } p = a^{(p-1)k_y + r} \text{ mod } p$ (fact about congruency)\\
	$(a^{(p-1)})^{k_x} \text{ mod } p * a^r \text{ mod } p = (a^{(p-1)})^{k_x} \text{ mod } p * a^r \text{ mod } p$ (property of exponents and proof from HW1 that the order of 'mod' does not matter)\\
	$1 * a^r \text{ mod } p = 1 * a^r \text{ mod } p$ (by Fermat's little theorm)\\
	$a^r \equiv a^r$\\ (fact about congruency)\\
	$r = x \text{ mod } p-1 = y \text{ mod } p-1$ (by definition of 'mod' and our premise)\\
	Therefore, if $x \equiv y$ (mod $p-1$) then for any $a, a^x \equiv a^y$ (mod $p$)\\
	\qed

	\item[(b)]
	Thm: if $g$ is a generator, then $g^x \equiv 1$ if and only if $(p-1) \mid x$\\
	\newline
	Proof: Assume $g^x \equiv 1$\\
	$g^{p-1} \equiv 1$ ($g \in Z^*_p$ by def'n of generator, Fermat's Little Theorm)\\
	$x = p-1$ (substitution)\\
	Therefore, $(p-1) \mid x$ (definition of divides)\\
	\newline
	Now, assume $(p-1) \mid x$\\
	Let $r = x$ (mod $p-1$) (Def'n of 'mod')\\
	$r = 0$ (def'n of divides)\\
	Consider $g^r$ (mod $p$)\\
	$g^{x \text{ mod } p-1}$ (substitution)\\
	$g^0 = 1$ (becasue $(p-1) \mid x$)\\
	Therefore $g^x \equiv 1$\\
	\qed

	\item[(c)]
	Thm: if $g$ is a generator, and $g^x \equiv g^y$ then $x \equiv y$ (mod $p-1$)\\
	\newline
	Proof: Let's assume $g^x \equiv g^y$ and $x \not\equiv y$ (mod $p-1$)\\
	$x \text{ mod } p-1 \neq y \text{ mod } p-1$ (Fact of congruency)\\
	$r_x \neq r_y$ (definition of mod)\\
	This means that $\exists_{r_x,r_y}$ $r_xr_y \in (1,...,p-1), r_x \neq r_y$ and $g^{r_x} \equiv g^{r_y}$\\
	However, $g$ is a generator, which means that each element in $(1,...,p-1)$ maps to a distinct element in $(1,...,p-1)$ (def'n of generator)\\
	Therefore, $g^{r_x} \not\equiv g^{r_y}$ which means $g^x \not\equiv g^y$\\
	This is a contradiction, therefore the statement must be true
	\qed

	\item[(d)]
	Thm: If $g$ is a generator, and $a = g^x$ (mod $p$), and $x$ is even, then $a$ has a square root modulo $p$\\
	\newline
	Proof: Because $x$ is even, we can rewrite it as $2y$ where $y$ is also a number in \{1,...,p-1\}\\
	$a = g^{2y}$ mod $p$\\
	$a = (g^y \text{ mod } p) * (g^y \text{ mod } p)$ (Splitting exponents with like bases)\\
	Because $g$ is a generator, we know that $g^y \in \{1,...,p-1\}$ and $g^y \not\equiv g^x$ (def'n of generator)\\
	Therefore $g^y$ is the square root of $a$ (Knowledge of square roots)\\
	\qed
	\newline
	Thm: if $a$ has a square root modulo then $x$ is even\\
	\newline
	Proof: Let's represent $a$ as a generator $g$ raised to some $x$ mod p.  $a = g^x$ mod $p$\\
	Let's also assume that $x$ is odd\\
	$g^x \equiv g^y * g^y$ (because we assume that $a$ has a square root)\\
	$g^x \equiv g^{2y}$\\
	This means that $x = 2y$\\
	This is a contradiction, because we assumed $x$ was even\\
	Therefore, if $a$ has a square root, then $x$ must be even.
	\qed

	\item[(e)]
	Thm: If $a$ is a square, then $a^{\frac{p-1}{2}} \equiv 1$\\
	\newline
	Proof: Let's assume there is a generator $g$ such that $g^x \equiv a$\\
	We know that $x$ must be even (by the previous part)\\
	$x = 2y$ for some $y \in \{1,...,p-1\}$\\
	$a \equiv g^{2y}$\\
	Now consider $(g^{2y})^{\frac{p-1}{2}}$\\
	$g^{y(p-1)}$ (2's cancel)\\
	Because $(p-1) \mid y(p-1)$ we know that $g^{y(p-1)} \equiv 1$ (proof from (b))\\
	Therefore, $a^{\frac{p-1}{2}} \equiv 1$\\
	\qed
	\newline
	Thm: If $a$ is non-square, then $a^{\frac{p-1}{2}} \not\equiv 1$\\
	\newline
	Proof: Let's assume there is a generator $g$ such that $g^x \equiv a$\\
	We know that $x$ must be odd (from proof (d))\\
	Now consider $(g^{x})^{\frac{p-1}{2}}$\\
	$g^{\frac{x}{2}(p-1)}$ (using rules of exponents\\
	$(p-1) \nmid \frac{x}{2}(p-1)$ therefore, $a^{\frac{p-1}{2}} \not\equiv 1$\\
	\qed
	
	\item[(f)]
	Thm: If $(g^x)^2 \equiv a$ then $(g^{x + (p-1)/2})^2 \equiv a$\\
	\newline
	Proof: We can rewrite $a$ as $g^{2x}$ (Rules of exponents)\\
	Now let's rewrite the latter expression:\\
	$g^{2(x + (p-1)/2)} \equiv g^{2x} * g^{p-1}$ (Rules of exponents)\\
	This can be rewritten as $g*{2x} * 1$ (by Fermat's Little Theorm)\\
	Therefore, $a \equiv (g^{x + (p-1)/2})^2$\\
	\qed
	\newline
	Thm: $g^{(p-1)/2} \equiv -1$\\
	\newline
	Proof: Consider $(g^{(p-1)/2})^2$\\
	$g^{p-1} \equiv 1$  (Rules of exponents and Fermat's Little Therom$\\
	Therefore, we know that $g^{(p-1)/2}$ is the square root of 1\\
	We know that $-g^x \equiv g^{x + (p-1)/2}$ (From the facts stated)\\
	




\end{enumerate}


\noindent\hrulefill


\section*{References}

None

\end{document} 