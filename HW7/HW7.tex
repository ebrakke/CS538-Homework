\documentclass[11pt]{article}
\usepackage{amssymb,amsmath,amsthm,graphicx}
\usepackage{fancyhdr}

\def\shownotes{1}   % set 1 for version with author notes
                    % set 0 for no notes



%uncomment to get hyperlinks
%\usepackage{hyperref}

%%%%%%%%%%%%%%%%%%%%%%%%%%%%%%%%%%%%%%%%%%%%%%%%%%%%%%%%%%%%%%
%Some macros (you can ignore everything until "end of macros")

\topmargin 0pt \advance \topmargin by -\headheight \advance
\topmargin by -\headsep

\textheight 8.9in

\oddsidemargin 0pt \evensidemargin \oddsidemargin \marginparwidth
0.5in

\textwidth 6.5in

%%%%%%

\providecommand{\vs}{vs. }
\providecommand{\ie}{\emph{i.e.,} }
\providecommand{\eg}{\emph{e.g.,} }
\providecommand{\cf}{\emph{cf.,} }
\providecommand{\etc}{\emph{etc.} }

\newcommand{\getsr}{\gets_{\mbox{\tiny R}}}
\newcommand{\bits}{\{0,1\}}
\newcommand{\bit}{\{0,1\}}
\newcommand{\Ex}{\mathbb{E}}
\newcommand{\eqdef}{\stackrel{def}{=}}
\newcommand{\To}{\rightarrow}
\newcommand{\e}{\epsilon}
\newcommand{\R}{\mathbb{R}}
\newcommand{\N}{\mathbb{N}}
\newcommand{\Gen}{\mathsf{Gen}}
\newcommand{\Enc}{\mathsf{Enc}}
\newcommand{\Dec}{\mathsf{Dec}}
\newcommand{\Sign}{\mathsf{Sign}}
\newcommand{\Ver}{\mathsf{Ver}}

\providecommand{\mypara}[1]{\smallskip\noindent\emph{#1} }
\providecommand{\myparab}[1]{\smallskip\noindent\textbf{#1} }
\providecommand{\myparasc}[1]{\smallskip\noindent\textsc{#1} }
\providecommand{\para}{\smallskip\noindent}


\newtheorem{theorem}{Theorem}
\theoremstyle{definition}
\newtheorem{ex}{Exercise}
\newtheorem{definition}{Definition}

%%%%%%%  Author Notes %%%%%%%d
%
\ifnum\shownotes=1
\newcommand{\authnote}[2]{{ $\ll$\textsf{\footnotesize #1 notes: #2}$\gg$}}
\else
\newcommand{\authnote}[2]{}
\fi
\newcommand{\Snote}[1]{{\authnote{Solution}{#1}}}
\newcommand{\Inote}[1]{{\authnote{Solution}{#1}}}
\newcommand{\Ichanged}[1]{{\authnote{Changed}{#1}}}
%%%%%%%%%%%%%%%%%%%%%%%%%%%%%%%%%

\newcommand{\VAR}{\mathrm{VAR}}



% end of macros
%%%%%%%%%%%%%%%%%%%%%%%%%%%%%%%%%%%%%%%%%%%%%%%%%%%%%%%%%%%%%%


% page counting, header/footer
\usepackage{fancyhdr}
\pagestyle{fancy}
\lhead{\footnotesize \parbox{11cm}{CS538, Boston University, Fall 2015} }
\rhead{Erik Brakke}
\renewcommand{\headheight}{24pt}

\begin{document}

\title{Homework 7}
\author{Erik Brakke}
\maketitle

\thispagestyle{fancy}

\myparab{Collaborators: }  Alison Kendler.
 
 
\section*{Answer 1}
We have $m, \sigma'$ and $(N, e)$ we can do the following:\\
$\sigma'^e = h^{de}$ mod $q$\\
And we also know that $\sigma'^e \neq h^{de}$ mod $p$ (Because we found this number with $q$ and $c$)\\
Thus, we know that $\sigma'^e = h^{de} + xq$\\
We also know that know that $H(m) = h^{de}$ mod $N$\\
We can figure out $H(m)$ because we have $m$ and $H$ is public\\
We can also do the following $\sigma'^e - H(m)$ and we are left with $xq$ (Because $\sigma'$ is still a number mod $N$)\\
We also know that $xq$ is a non trival multiple of $q$ mod $N$ because $\sigma'^e \neq h^{de}$ mod $p$\\
Thus we can do $gcd(N, \sigma'^e - H(m))$ to factor $N$\\

\section*{Answer 2}
Suppose we have a function $f$ that given a message $m \in M$ s.t. $|M| = 20$ will output 3 distinct numbers in $\{0,1,2,3,4,5\}$.  This function is publicly known.  We can now use this to extend Lamport's signing algorithm to all 20 messages.  Because there are $6$ keys in $SK$, and we are choosing 3 of them, we have 6 choose 3, or 20, possible combination that we can choose values from $SK$.\\  
\newline
The signer generates $SK$ and $PK$ just as in Lamports using a OWF.  The signer then makes $PK$ publicly available (PK is index 0-5).  The, given a message $m$, the signer runs $f(m) = (i_0,i_1,i_2)$\\
The signer then sends the message $m$ along with $SK_{i_0,i_1,i_2}$\\
To verify, the recipient must run $f(m) = j_0,j_1,j_2$ and very that $OWF(SK_{i_0}) = PK_{j_0}$ and repeat this for the other two positions\\
If each succeeds, then the message is valid.\\
The signature cannot be forged because $f$ only reveals the positions, not the actual $SK$ values.  And because this is one time, a forger cannot use prior knowledge of the SK to forge a new signature.\\
Finally, forgery depends on the one way function.\\
The only knowledge the forger has is $f$ which gives positions, and $PK$.  He cannot modify either of these are they are publicly known.  Also, because each position is unique and each combination is unique, the forger cannot use the SK of a message to forge the SK of another message because he will always be missing one value.\\
Thus, he must reverse the OWF on the public key to forge a message\\

\section*{Answer 3}
Given $n$ values and a hash function $H$:\\
Start at $x_1$.  
Compute $H(x_i,x_{i+1})$ and store in the parent\\
If the sibling of the parent has no value, compute and store in the sibling\\
When the sibling has a hash value, take $H(self, sibling)$ and store it in the parent.  Forget the hash of self and sibling and move to the parent\\
Continue doing the above steps until you have reached the root.
At every parent, you only have to store the value of self + the max number of values stored at one time to compute the sibling\\
Thus, at each level, you store 1 + the max number of values stored at one time to compute the sibling\\
Thus, you only have to store hash values = the depth of the tree (If you can instantly forget two hash values once the parent is computed)\\
The depth of the tree is $log n$\\
You only have to store at most $\lceil log n \rceil$\\




\noindent\hrulefill


\section*{References}

None

\end{document} 