\documentclass[11pt]{article}
\usepackage{amssymb,amsmath,amsthm,graphicx}
\usepackage{fancyhdr}

\def\shownotes{1}   % set 1 for version with author notes
                    % set 0 for no notes



%uncomment to get hyperlinks
%\usepackage{hyperref}

%%%%%%%%%%%%%%%%%%%%%%%%%%%%%%%%%%%%%%%%%%%%%%%%%%%%%%%%%%%%%%
%Some macros (you can ignore everything until "end of macros")

\topmargin 0pt \advance \topmargin by -\headheight \advance
\topmargin by -\headsep

\textheight 8.9in

\oddsidemargin 0pt \evensidemargin \oddsidemargin \marginparwidth
0.5in

\textwidth 6.5in

%%%%%%

\providecommand{\vs}{vs. }
\providecommand{\ie}{\emph{i.e.,} }
\providecommand{\eg}{\emph{e.g.,} }
\providecommand{\cf}{\emph{cf.,} }
\providecommand{\etc}{\emph{etc.} }

\newcommand{\getsr}{\gets_{\mbox{\tiny R}}}
\newcommand{\bits}{\{0,1\}}
\newcommand{\bit}{\{0,1\}}
\newcommand{\Ex}{\mathbb{E}}
\newcommand{\eqdef}{\stackrel{def}{=}}
\newcommand{\To}{\rightarrow}
\newcommand{\e}{\epsilon}
\newcommand{\R}{\mathbb{R}}
\newcommand{\N}{\mathbb{N}}
\newcommand{\Gen}{\mathsf{Gen}}
\newcommand{\Enc}{\mathsf{Enc}}
\newcommand{\Dec}{\mathsf{Dec}}
\newcommand{\Sign}{\mathsf{Sign}}
\newcommand{\Ver}{\mathsf{Ver}}

\providecommand{\mypara}[1]{\smallskip\noindent\emph{#1} }
\providecommand{\myparab}[1]{\smallskip\noindent\textbf{#1} }
\providecommand{\myparasc}[1]{\smallskip\noindent\textsc{#1} }
\providecommand{\para}{\smallskip\noindent}


\newtheorem{theorem}{Theorem}
\theoremstyle{definition}
\newtheorem{ex}{Exercise}
\newtheorem{definition}{Definition}

%%%%%%%  Author Notes %%%%%%%d
%
\ifnum\shownotes=1
\newcommand{\authnote}[2]{{ $\ll$\textsf{\footnotesize #1 notes: #2}$\gg$}}
\else
\newcommand{\authnote}[2]{}
\fi
\newcommand{\Snote}[1]{{\authnote{Solution}{#1}}}
\newcommand{\Inote}[1]{{\authnote{Solution}{#1}}}
\newcommand{\Ichanged}[1]{{\authnote{Changed}{#1}}}
%%%%%%%%%%%%%%%%%%%%%%%%%%%%%%%%%

\newcommand{\VAR}{\mathrm{VAR}}



% end of macros
%%%%%%%%%%%%%%%%%%%%%%%%%%%%%%%%%%%%%%%%%%%%%%%%%%%%%%%%%%%%%%


% page counting, header/footer
\usepackage{fancyhdr}
\pagestyle{fancy}
\lhead{\footnotesize \parbox{11cm}{CS538, Boston University, Fall 2015} }
\rhead{Erik Brakke}
\renewcommand{\headheight}{24pt}

\begin{document}

\title{Homework 8}
\author{Erik Brakke}
\maketitle

\thispagestyle{fancy}

\myparab{Collaborators: }
 
 
\section*{Answer 1}
\begin{enumerate}
	\item[(a)] First, let's assume we have an experiment where $D$ sends two messages $m_0,m_1$ and gets back an encryption of one of these messages where $b$ is chosen at random.  $b$ is the message that is encrypted.\\
We know that $\Pr[\text{D outputs b} | exp-b] = 1/2 + neg$\\
That is, $D$ does not have any advantage by seeing $(y,d)$ in choosing $b$\\
Let's assume this is not the case\\
Given a $y_b,d_b$ from $m_b$ where $m_b$ is one of $m_0,m_1$ chosen by $D$, $\Pr[\text{D outputs b}|exp-b] = 1/2 + \epsilon$ Where $\epsilon$ is some non-negligible probability\\
This means that $D$ is able to distinguish $(y_0,d_0)$ from $(y_1,d_1)$ with some non-negligible advantage $\epsilon$\\
Let the event $E$ be the event that $D$ queries $H$ on $r = f^{-1}(y)$
If $\bar{E}$, then that means $D$ never asked for $H(r)$.\\
Because $d_0$ is just $p \oplus m$, and $p$ is $H(r)$, $D$ needs to know $H(r)$ to have any kind of advantage on distinguishing\\
Thus, if $\bar{E}$, then $D$ does not have any advantage on distinguishing regardless of $b$\\
Because he has no advantage, then $Pr[\text{Guessing correctly} | \bar{E}] = 1/2$
If we let $S$ be the event that $D$ guesses $b$ correctly, we have $\Pr[S] = \Pr[S|E]\Pr[E] + \Pr[S|\bar{E}]\Pr[\bar{E}]$\\
We know that $Pr[S] = 1/2 + \epsilon$\\
$1/2 + \epsilon = \Pr[S|E]\Pr[E] + 1/2(1 - \Pr[E])$\\
$1/2 + \epsilon = \Pr[E] + 1/2 - \Pr[E]/2$ (Because if $E$, then $D$ will distinguish)\\
$2*\epsilon \le \Pr[E]$

	\item[(b)] Let's use $D$ to build a function $R$ that can reverse the OWF $f$\\
We will give $R$ the inputs $PK, y$ and will expect that $R$ can use $D$ to output $f^{-1}(y)$ with non-negligible probability\\
Let's assume that $F$ will has $H$ for the hash of 2 values $r_0, r_1$ before sending $m_0,m_1$ to $Enc$ (because he will need these to distinguish)
$R$ must create $H(a_j)$ as such:\\
If $a_j$ has been queried before, then output the same value $s_j$.  If not then output a random $s \in D_i$ and store the value\\
$R$ must create $Enc(m_j)$ in the following way:\\
Find $a_j, s_j$ and output $(f_i(a_j), s_j \oplus m_j)$ where $f$ is the OWF\\
When $D$ is ready to guess, he will send $m_0$ and $m_1$ to $Enc$\\
$Enc$ will choose a message $m_b$ and return $(y, H(r_b) \oplus m_b)$
Now $R$ can run the OWF $f$ on $r_0$ and $r_1$ to see which one is $f^{-1}(y)$
This is a contradiction, because we assumed $f$ was a OWF, therefore this encryption scheme is polynomially secure\\
\qed

\end{enumerate}

\section*{Answer 2}
Given a random oracle $H$ we can construct another random oracle $H'$ to hash to arbitrary lengths by doing the following:\\
$h_1 = H(n || s)$\\
$h_2 = H(n || h_1)$\\
...\\
$h_i = H(n || h_{i-1})$\\
Then you have $h = h_1||h_2||...||h_i$ and output $n$ bits $h_1...h_n$\\
In this case $i = \lceil n/l \rceil$\\
The outputs will always be independent of eachother because either $s$ will be different or $n$ will be different  between two pains $(n,s)$\\
Because we take $H(n||s)$ first, we are guarenteed that this will be unique (i.e. no collisions with greater than negligible probability)\\
This means that $h_2...h_i$ will also be unique, so concatonating them together will give us indepent outputs

\section*{Answer 3}
The adversary $A$ will generate $m_0,m_1$\\
He will then receive $(c = g^{ab} * m_b \text{ mod } p, g^b \text{ mod } p)$\\
$A$ can then compute $(g^{ab} * m_b * g^b \text{ mod } p, g^b \text { mod } p$ and send this to $Dec$\\
$A$ will get $m_b * g^b \text{ mod } p$ back\\
He can then multiply by the inverse of $g^b \text{ mod } p$ and recover $m_b$\\
Then he will be able to distinguish
\noindent\hrulefill


\section*{References}

None

\end{document} 